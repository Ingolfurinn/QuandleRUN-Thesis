
\documentclass{Thesis/mcom-l}




\begin{document}
\title{Abstract - Seminar - May 26th}
\author{Niccolò Carrivale - s1035576}

\maketitle

\begin{abstract}[\textsf{English version}]
    Computer algebra involves developing algorithms to carry out exact computations involving algebraic structures. Quandles are algebraic structures that can benefit from the help of computational tools and attention from the community, much like groups. This project introduces a modern tool able to work with quandles: congruences, subuniverses, isomorphisms, and monomorphisms, among others; built on the solid mathematical foundations of the most recent developments in Quandle Theory, and Universal Algebra as well as computational tools that withstood the test of time such as Magma and Rig and modern ones such as CREAM, it is able to obtain useful results at a competitive speed. As an example, it will be shown how to compute and visualise connections among connected quandles.
    A new algorithm for computing the congruences of connected quandles, based on the work of dr. Marco Bonatto, will be introduced.
\end{abstract}
\begin{abstract}[\textsf{Italian Version}]
    L'algebra computazionale si occupa di sviluppare algoritmi esatti per facilitare le elaborazioni simboliche con strutture algebriche. I quandles sono strutture algebriche che possono beneficiare dall'aiuto di strumenti computazionali e dall'attenzione della comunità, come i gruppi. Questo progetto presenta uno strumento moderno in grado di lavorare con i quandles: congruenze, subuniversi, isomorfismi e monomorfismi, tra le altre cose; siluppato sulle solide fondamenta matematiche dei più recenti sviluppi nella teoria dei quandles e algebra universale così come strumenti computazionali che hanno resistito alla prova del tempo come Magma e Rig ma anche moderni come CREAM, è in grado di ottenere risultati utili a velocità competitive. Come esempio, verrà mostrato come calcolare e visualizzare le connessioni tra quandle connessi. Sarà introdotto un algoritmo all'avanguardia per calcolare le congruenze dei quandles connessi, basato sul lavoro di dr. Marco Bonatto.
\end{abstract}

\end{document}