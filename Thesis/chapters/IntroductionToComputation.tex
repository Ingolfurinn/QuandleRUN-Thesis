\documentclass{../mcom-l}

\begin{document}
The first step in manipulating something using a computer is the creation of a suitable representation: similar to the idea of adjancency matrices to represent graphs. In \cite{ho2005matrices}, Benita Ho and Sam Nelson introduce the idea of \emph{quandle matrix}, that is, essentially, the quandle operation table:
\begin{quote}
    Let $\mathbf{A}=(\{x_1, x_2, \dots, x_n\}, \cdot)$ be a finite quandle with $n$ elements. \newline 
    The quandle matrix $M_\algebra{A}$ of $\algebra{A}$ is:
    \[M_\algebra{A} = \begin{bmatrix}
 x_1 \cdot x_1 & x_1 \cdot x_2 & \dots & x_1 \cdot x_n \\
 x_2 \cdot x_1 & x_2 \cdot x_2 & \dots & x_2 \cdot x_n\\
 \vdots & \vdots & \ddots & \vdots \\
 x_n \cdot x_1 & x_n \cdot x_2 & \dots & x_n \cdot x_n
\end{bmatrix}\]
The entry in row $i$ and column $j$ is $x_i \cdot x_j$\cite{ho2005matrices}.\newline
\end{quote}
Without loss of generality, one can restrict discussions about quandle matrix to \emph{integral quandle matricx}, that is, the quandle matrix of a quandle $\algebra{A}$ with underlying set $S =\{1,2,\dots,n\}$ and $M_\algebra{A} = \begin{bmatrix} m_{ij}\end{bmatrix}$. Given the quandle matrix of any quandle $\algebra{A}=(\{x_1, x_2, \dots, x_n\}, \cdot)$, one can obtain an integral quandle matrix by mapping $x_i$ to $i$, for $1 \leq i \leq n$\cite{ho2005matrices}. This determines clear methods to verify whether a given quandle matrix actually represents a quandle. However, this also takes some flexibility away from the scholar.\newline
\Software is the first tool trying to relax these constraints by asking the user for a standard representation of the underlying set of the quandle, against which the quandle matrix is then checked. However, this comes at a complexity cost. 

\subsection{Creation and verification of a quandle}
There are several known quandle construction already implemented in \Software and even more can easily be independently developed given the framework available in \Software.

\begin{example}\cite{lopes2006finite}
\textbf{Dihedral quandle} of order $n$, denoted $\algebra{R_n}$:
\[\algebra{R_n} = (\{1,2,\dots,n\}, x \cdot y = 2x-y\mod{n})\]
For example, $\algebra{R_n}$:
\[\algebra{R_n} = (\{1,2,3\}, x \cdot y = 2x-y\mod{3}) \quad M_\algebra{R_n}=\begin{bmatrix}
 1 & 3 & 2\\
 3 & 2 & 1\\
 2 & 1 & 3
\end{bmatrix}\]
\begin{lstlisting}[language=C]
R_3 := DihedralQuandle({1 .. 3});
R_3 := DihedralQuandle({7 .. 9});
\end{lstlisting}
\end{example}

\begin{example}\cite{lopes2006finite}
\textbf{Trivial quandle} of order $n$, denoted $\algebra{T_n}$:
\[\algebra{T_n} = (\{1,2,\dots,n\}, x \cdot y = y)\]
For example, $\algebra{T_n}$:
\[\algebra{T_n} = (\{1,2,3\}, x \cdot y = 2x-y) \quad M_\algebra{T_n}=\begin{bmatrix}
 1 & 2 & 3\\
 1 & 2 & 3\\
 1 & 2 & 3
\end{bmatrix}\]
\begin{lstlisting}[language=C]
T_3 := TrivialQuandle({1 .. 3});
\end{lstlisting}
\end{example}
At this point in time, \Software only works with integral labels for the elements of a quandle.\newline A way to create a quandle quandle $\algebra{A}=(S,\cdot)$ ``from scratch" is by providing \Software with
the quandle matrix $M_\algebra{A}$.
\begin{example}\textcolor{white}{skip}\newline
\begin{lstlisting}[language=C]
QuandleFM([[7,9,8],[9,8,7], [8,7,9]], false); 
// false indicates that it should not expect a standard set {1..n}
\end{lstlisting}
\end{example}
Another way to create a quandle $\algebra{A}=(S,\cdot)$ ``from scratch" is by providing \Software with an integral representation $S_\mathbb{N}$ of the underlying set $S$ and a map $\cdot : S_{\mathbb{N}}^2 \to \mathbb{N}$. Such a function does not seem to be available in any existing tool. 

\TODO{WRITE EXAMPLE}

Other constructions, perhaps more used by professional mathematicians, arise from groups. 

\begin{example}
\cite{bianco2021connected}
\textbf{Coset Quandle}. Let G be a group, $f \in \text{Aut}(G)$ and \linebreak $H \leq \{x \in G : f(x) = x\}$.\newline
The left coset $G/H$ and the operation $xH \cdot yH = xf(x^{-1}y)H$ form a coset quandle, denoted by  $\mathcal{Q}_{\text{Hom}}(G,H,f)$.
\end{example}

\end{document}

