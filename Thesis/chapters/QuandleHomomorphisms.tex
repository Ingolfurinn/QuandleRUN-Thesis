\documentclass{../mcom-l}

\begin{document}



\begin{definition}\cite{burris1981course} 
Let $\mathbf{A} = (S, \cdot)$ and $\mathbf{B}=(S',\star)$ be two groupoids.\linebreak Let $\mathcal{H} : \mathbf{A} \to \mathbf{B}$.\newline
If 
\[\forall x,y \in \mathbf{A} \quad \mathcal{H}(x\cdot y)= \mathcal{H}(x)\star \mathcal{H}(y)\]
then $\mathcal{H}$ is a \emph{homomorphism}.
\end{definition}

\begin{definition}\label{monomorphismDefinition}\cite{burris1981course} 
Let \textbf{A} and \textbf{B} be two groupoids. Let $\mathcal{H} : \mathbf{A} \to \mathbf{B}$ be a homomorphism. If
\[\forall x, y \in \mathbf{A} \quad \mathcal{H}(y) = \mathcal{H}(x) \implies x = y,\]
that is, $\mathcal{H}$ is injective, $\mathcal{H}$ is called a \emph{monomorphism}, or an \emph{embedding} of \textbf{A} into \textbf{B}. $\mathcal{H}(\mathbf{A})$ indicates \emph{subalgebra} of \textbf{B}, indicated $\mathcal{H}(\mathbf{A}) \leq \mathbf{B}$.\newline\newline
If there exists a monomorphism $\mathcal{H}: \mathbf{A} \to \mathbf{B}$, it is denoted by $\mathbf{A} \hookrightarrow \mathbf{B}$.
\end{definition}
% \begin{example}
% \[\begin{bmatrix}
% b& f & c \\
% f & c & b\\
% c & b& f
% \end{bmatrix}
%  \overset{\phi}{\hookrightarrow}
% \begin{bmatrix}
%  a& b& e& f& c& d \\
%      a& b & f & e& d& c \\
%      e& f& c& d& a& b \\
%      f& e& c& d& b& a \\
%      c& d& a& b& e& f \\
%      d& c& b& a& e& f 
% \end{bmatrix}\]
% \end{example}
\begin{definition}\cite{burris1981course} 
Let \textbf{A} and \textbf{B} be two groupoids. Let $\mathcal{H} : \mathbf{A} \to \mathbf{B}$ be a homomorphism. If
\[\forall x \in \mathbf{B} \quad \exists y \in \mathbf{A} \quad  \mathcal{H}(y)=x,\]
that is, $\mathcal{H}$ is surjective, $\mathcal{H}$ is called an \emph{epimorphism}; if there is an epimorphism $\mathcal{H}: \mathbf{A}\to \mathbf{B}$, it is denoted by $\mathbf{A}\twoheadrightarrow \mathbf{B}$.
\end{definition}

\begin{definition}\cite{burris1981course} 
Let \textbf{A} and \textbf{B} be two groupoids. Let $\mathcal{H} : \mathbf{A} \to \mathbf{B}$ be a homomorphism. If $\mathcal{H}$ is injective and surjective, that is, $\mathcal{H}$ is bijective, $\mathcal{H}$ is called an \emph{isomorphism}. If there exists an isomorphism $\mathcal{H}: \mathbf{A} \to \mathbf{B}$, it is indicated $\mathbf{A} \cong \mathbf{B}$.\newline\newline
In case $\mathbf{A}=\mathbf{B}$, $\mathcal{H}$ is called an \emph{automorphism}.
\end{definition}

\begin{theorem}\label{isoismono}
Let $\mathbf{A}=(S,\cdot)$ and $\mathbf{B}=(S', \star)$ be finite groupoids.
\[ |S|=|S'| \land \mathbf{A} \hookrightarrow \mathbf{B} \implies \mathbf{A} \cong \mathbf{B}\]
\begin{proof}
Theorem 17.7 in \cite{warner1990modern}.
\end{proof}
\end{theorem}

% \noindent Definition \ref{monomorphismDefinition} delineates a clear decision procedure to the problem \textit{Is a defined algebra \textbf{A} a subalgebra of a defined algebra \textbf{B}?} It is sufficient to prove the existence of a monomorphism $\mathcal{M}: \mathbf{A}\to \mathbf{B}$.\newline
% The way this is approached in \Software is by actually finding a monomorphism.


\end{document}

