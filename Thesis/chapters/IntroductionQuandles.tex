\documentclass{../mcom-l}

\begin{document}




% \begin{definition}
% \cite{bourbaki2013theory} %Pages 64,65
% Let $I,Y$ be sets. Let $f : I \to Y$ be a function.\newline
% For $i \in I$, the unique value of $i$ under $f$ can be denoted by $f(i)$(or $f_i$).\newline The functional graph $G_f = \{( i, Y) \in I \times Y \mid y = f(i) = f_i\} \subset I \times Y$ is called a \emph{family}.\newline The set $\mathsf{ran}(f)=\{ f_i \mid i \in I\}$ can be called \emph{indexed set}, and, $I$ is called \emph{index set}.

% \end{definition}

\begin{definition}\textcolor{white}{line}\newline
Let $S \neq \emptyset$ and $n \in \mathbb{N}$.\newline
For $n=0$, $S^n = \{ \emptyset \}$, and, for $n>0$, $S^n = \underbrace{S \times S \times \dots \times S}^{~n~\text{times}}$.\newline
A \emph{unary} operation on S is a function $f : S \to S$.\newline
A \emph{binary} operation on S is a function $f : S^2 \to S$.\newline
A \emph{n-nary} operation on S is a function $f : S^n \to S$.
\end{definition}

% \begin{definition}
% \cite{burris1981course} 
% An algebra $\mathbf{A}$ consists of a set of elements $S\neq \emptyset$ and a set of operations on $S$, $F$.\newline
% Often, when $F$ is finite, $F =\{f_1, f_2, \dots, f_n\}$, one can indicate $\mathbf{A}$ with \linebreak $( S, f_1, f_2, \dots, f_n )$.
% \end{definition}
This work will only discuss finite algebras.
% \begin{definition}
% \cite{burris1981course}
% A \emph{type} of algebras is a set $\mathfrak{F}$ of function symbols.\newline
% Each $f \in \mathfrak{F}$ is assigned a $n \in \mathbb{N}$, indicating its arity.\newline
% $\mathfrak{F}_n$ denotes the subset of $n$-ary function symbols.
% \end{definition}

% \begin{definition}
% \cite{burris1981course} %conflict between burris and denecke
% An algebra $\mathbf{A}=(S,  F)$  of type $\mathfrak{F}$ is a $2$-tuple with $S\neq \emptyset$ and $F$ being a family of operations on $S$ with index set $\mathfrak{F}$.
% $F$ assigns a $n$-ary operation $f^{\mathbf{A}}$ on S, called \emph{fundamental operation}, to each function symbol in $\mathfrak{F}$.
% Often, when $\mathfrak{F}$ is finite, $\mathfrak{F} =\{f_1, f_2, \dots, f_n\}$, one can represent $( S, F)$ with $( S, f_1, f_2, \dots, f_n)$.
% \end{definition}
\begin{definition}\textcolor{white}{line}
% An algebra $\mathbf{A} = ( S, \cdot, \backslash )$ is a \emph{groupoid} if $\cdot$ is a binary operation.\newline\newline
\begin{itemize}
    \item Let $S$ be a set endowed with two binary operations $\cdot$ and $\backslash$ such that \[\forall x \in S\quad L_x : S \to S \hspace{1em} y \mapsto x \cdot y \hspace{1em} L_x~\text{is bijective}\hspace{1em}\text{(left multiplication)},\]
\[x\backslash y = L_x^{-1}(y),\hspace{2em}\text{for}~x,y \in S\hspace{1em}\text{(left division)}\]then $\algebra{A} = (S, \cdot, \backslash)$, or, given the definition of left division, $\algebra{A} = (S, \cdot)$ is called a \emph{left quasigroup}.  \cite{bonatto2022medial}
\item Let $\mathbf{A} = ( S, \cdot )$ be a left quasigroup. 
If 
\[\forall x,y,z \in S \quad x \cdot (y \cdot z) = (x \cdot y) \cdot (x \cdot z) \hspace{2em}\text{(left self-distributivity),}\]
then $\mathbf{A}$ is called a \emph{rack}.
\item Let $\mathbf{A} = ( S, \cdot )$ be a rack.
If 
\[\forall x \in S \quad x \cdot x = x \hspace{2em}\text{(idempotency),}\]
then $\mathbf{R}$ is called a \emph{quandle}.
\end{itemize}
\end{definition}

\begin{definition}\cite{burris1981course}\newline
Let $\mathbf{A} = (S, \cdot)$ be a grupoid. Let $S'\subseteq S$. If \[\forall x,y \in S'\quad x\cdot y \in S',\]
that is, $S'$ is closed under the operation of $\mathbf{A}$, then $S'$ is called a \emph{subuniverse} of $\mathbf{A}$.\newline\newline
Let $\mathbf{A} = (S, \cdot), \mathbf{B} = (S', \star)$ be two groupoids. If
\[S' \subseteq S \quad\land\quad \forall x,y \in \mathbf{B}\quad x\star y = x\cdot y\]
then $\mathbf{B}$ is a \emph{subalgebra} of $\mathbf{A}$.

\end{definition}


\end{document} 

