 \begin{frame}{QuandleRUN}
QuandleRUN is a library of algorithms focusing on finite quandles that I am developing under the supervision of\newline\newline
dr. Wieb Bosma - Computer Algebra - Radboud University \newline
dr. Marco Bonatto - Non-Associative Algebra - University of Ferrara\newline\newline
QuandleRUN aims to expand the capabilities of the only existing package to work with \textbf{finite} quandles, Rig\footnote{L. Vendramin. Rig, a GAP package for racks, quandles and Nichols algebras. Available at \href{https://github.com/gap-packages/rig/}{https://github.com/gap-packages/rig/} \newline}, while providing a tool useful to researchers.  
\end{frame} 

\begin{frame}{Question}

    \textbf{How to develop a modern computational tool for Quandle Theory? \newline
    \begin{itemize}
        \item Which algorithms should such a computational tool have?
        \item Does the theory offer faster ways of doing things?
        \item How does this tool compare to other existing tools?
    \end{itemize}}


\end{frame} 






\begin{frame}{Results\&Conclusions}
    \begin{itemize}
        \item I have developed QuandleRUN, a real module for the CAS \textsc{MAGMA}.
        \item I have managed to implement several important algorithms such as those to compute all subquandles of a quandles, all of its congruences, its Inner Automorphism Group and many more. 
        \item I have "translated" the largest database of connected quandles from \textsc{GAP} into \textsc{MAGMA} and JSON.
        
        
        
    \end{itemize}
    
\end{frame}

\begin{frame}{Results\&Conclusions}
    \begin{itemize}
        
        
        \item Some of the algorithms I have developed are not available in any existing package.
        \item Some of the algorithms I have developed are faster than the standard algorithm for that task, \emph{in practice}.
        \item Recently developed theory played a key role in the development of the faster algorithms. 
        
        \item I have presented my tool at the Department of Mathematics and Computer Science of the University of Ferrara. 
    \end{itemize}
    
\end{frame}









