\documentclass{mcom-l}

\begin{document}

\noindent
Independently described by David Joyce in \cite{JOYCE198237} and Sergei V. Matveev in \cite{SergeiMatveevDisGrou}, Quandles are algebraic structures that play a role in several of the traditional sciences. In \cite{JOYCE198237}, David Joyce indicates quandles as a classifying invariant of knots which means that they play a role in what still is one of the fundamental problems of knot theory, the classification of knots - a computationally hard problem. Still in relation to knot theory, the quandle axioms are equivalent, in the context of Tietze transformations, to the three Reidemeister moves\cite{lie2algebras, nelson2005signed}. Furthermore, Quandles arise as set-theoretical solutions to the quantum Yang-Baxter equation in statistical mechanics\cite{BONATTO2021128}. Quandles are connected to spindles - a quandle is a spindle - which are a useful tool in studying the relation between a Lie Algebra and its Lie Group.\cite{lie2algebras}. Quandles can also be adapted to distinguish the topological types of proteins, thus obtaining \emph{bondles}\cite{adams2020knot}. This motivates the efforts in the study of quandles and the creation of computational tools that can help in such an endeavour. The greatest example of such a tool is \textsf{Rig}\cite{RiGapVendramin}, a package for the computer algebra system \textsf{GAP}\cite{GAPLinton}. Computer algebra involves developing algorithms to carry out \emph{exact} computations involving algebraic structures, such as groups, graphs and algebraic curves which play important roles in modern mathematics. Contrary to results of numerical methods which concern themselves with approximate solutions, computer algebraic procedures can help in solving decision problems such as \textit{Is a defined quandle Q connected?} or \textit{Is quandle Q' a subquandle of quandle Q?}. This work is interested in developing solid foundations on which future computational work on quandles can be built and does so by combining results from the mathematical theory of quandles and adapting existing computer algebraic methods.



\end{document}

%------------------------------------------------------------------------------
% End of mcom_sample.tex
%------------------------------------------------------------------------------